\chapter{Conclusões e Trabalho Futuro}

\section{Conclusões}

O trabalho desenvolvido permitiu conceber e implementar uma solução de monitorização e observabilidade para o projeto \textit{R2UT}, reforçando a visibilidade e a resiliência da sua infraestrutura de microserviços. A integração das ferramentas \textit{open source} \textit{Prometheus}, \textit{Grafana}, \textit{Loki}, \textit{Jaeger} e \textit{OpenTelemetry Collector} revelou-se eficaz para recolher, centralizar e correlacionar métricas, \textit{logs} e \textit{traces} de forma padronizada.

Durante o desenvolvimento, foram superados diversos desafios técnicos relacionados com a orquestração dos componentes, a instrumentação da aplicação e a gestão de recursos em ambiente \textit{cloud-native}. Estes obstáculos contribuíram para um melhor entendimento das boas práticas de observabilidade e consolidaram a robustez da solução final.  

Os resultados obtidos demonstram uma redução significativa dos tempos médios de deteção e resolução de incidentes (\textit{MTTD} e \textit{MTTR}), bem como uma melhoria da estabilidade operacional do sistema. Em síntese, foi validada a viabilidade de uma pilha de observabilidade escalável e aberta, totalmente baseada em tecnologias de utilização livre, alinhada com os objetivos do projeto \textit{R2UT}.

\section{Trabalho Futuro}

Como continuidade deste trabalho, prevê-se a expansão da solução de observabilidade através da integração de mecanismos de automação e inteligência artificial para análise preditiva de métricas e deteção de anomalias. A aplicação de técnicas de \textit{machine learning} poderá permitir a antecipação de falhas e a geração de alertas inteligentes com base em padrões históricos de comportamento.

Adicionalmente, pretende-se aprofundar a integração com ferramentas de orquestração em larga escala, avaliando o desempenho do sistema em ambientes \textit{multi-tenant} e cenários de elevada carga. Também se prevê a criação de painéis avançados de \textit{dashboards} e relatórios dinâmicos, bem como o estudo de políticas de \textit{autoscaling} e de recuperação automática de serviços.

Por fim, a disseminação dos resultados e a integração desta solução no ecossistema de produção da plataforma \textit{R2UT} representam uma oportunidade de validação real do seu impacto, contribuindo para uma infraestrutura mais inteligente, resiliente e observável.
