\chapter{Conclusões e Trabalho Futuro}

\section{Conclusões}

Nesta dissertação começámos por realizar uma análise crítica e aprofundada da literatura disponível sobre arquiteturas de microsserviços e sistemas de monitorização em ambientes distribuídos. Esse trabalho visou construir uma base teórica sólida que suportasse o desenvolvimento de uma plataforma de monitorização eficaz e alinhada com os objetivos do projeto R2UT, destacando-se o uso de ferramentas \textit{open-source} amplamente reconhecidas, como Prometheus, Grafana, ELK Stack e Jaeger.

A revisão bibliográfica incidiu particularmente nos desafios associados à monitorização de sistemas distribuídos dinâmicos, procurando identificar padrões de resiliência e boas práticas que assegurassem a fiabilidade, a escalabilidade e a eficiência operacional dos microsserviços. Foram igualmente aprofundados conceitos fundamentais como \textit{distributed tracing}, \textit{alerting} e observabilidade, analisando como estas técnicas podem ser combinadas para otimizar a gestão e operação de infraestruturas baseadas em microsserviços. A partir desta fundamentação teórica, foi possível estabelecer os requisitos funcionais e tecnológicos que orientaram a implementação prática.

No contexto do projeto R2UT, foi selecionado um conjunto de serviços representativos do ecossistema da plataforma, com o objetivo de demonstrar o funcionamento da solução num ambiente real e alinhado com o fluxo operacional do sistema. A escolha destes serviços permitiu validar a abordagem tomada, avaliar a capacidade de escalabilidade da solução e aferir a eficácia na deteção e investigação de falhas, garantindo simultaneamente a continuidade operacional da plataforma durante a fase de migração para Kubernetes.

A implementação de uma solução de monitorização baseada em OpenTelemetry, integrando ferramentas como Prometheus, Loki, Jaeger e Grafana, proporcionou uma visão completa e em tempo real do comportamento da aplicação distribuída. A arquitetura modular e escalável oferece flexibilidade, portabilidade e facilidade de manutenção. A correlação sinérgica entre \textit{logs}, métricas e \textit{traces}, potenciada pela padronização do OpenTelemetry, transformou a forma como a equipa de desenvolvimento e operações lida com problemas de desempenho, bem como permitiu a deteção e correção proativa de falhas e \textit{bottlenecks} no fluxo de execução dos microsserviços. A capacidade de navegar de um alerta de métrica diretamente para os \textit{traces} e \textit{logs} correspondentes em poucos segundos demonstra claramente o valor operacional desta abordagem.

Importa ainda referir que, durante o desenvolvimento, foi avaliada a adoção de \textit{zero-code instrumentation} para aplicações .NET, recorrendo a agentes automáticos de OpenTelemetry. Embora esta abordagem apresente vantagens em termos de rapidez de integração e eliminação de alterações diretas no código, verificaram-se limitações relevantes ao nível do controlo de granularidade, compatibilidade com bibliotecas utilizadas no ecossistema da aplicação e flexibilidade na definição de atributos específicos de negócio. Estas restrições levaram à opção por uma solução híbrida, baseada num pacote comum de instrumentação (\texttt{DTX.Base.Common}) e num método de extensão centralizado, garantindo simultaneamente padronização, flexibilidade e governância sobre os sinais de telemetria emitidos. Assim, a solução final alcançou um equilíbrio entre automatização e controlo fino da instrumentação, assegurando consistência técnica e alinhamento com as necessidades operacionais da plataforma.


Do ponto de vista prático, a solução revelou-se eficaz na redução do tempo médio de diagnóstico (\textit{MTTR}) e na identificação rápida de anomalias, quer ao nível da aplicação, quer ao nível da infraestrutura. Entre os principais benefícios observados destacam-se a padronização da telemetria entre serviços, a capacidade de correlação entre diferentes sinais e a redução da dependência de ferramentas proprietárias. Esta abordagem contribuiu igualmente para uma maior transparência no funcionamento interno dos serviços, permitindo apoiar decisões informadas em fases de desenvolvimento, testes e produção.

Contudo, a implementação também evidenciou desafios relevantes. A configuração inicial do \textit{pipeline} de telemetria requer conhecimento técnico especializado, nomeadamente na definição de \textit{pipelines} e recursos do OpenTelemetry Collector, de modo a evitar perdas de dados, degradação de desempenho ou consumo excessivo de memória. Adicionalmente, a instrumentação introduz algum \textit{overhead} computacional, pelo que se torna essencial aplicar boas práticas de controlo de cardinalidade, \textit{sampling} e políticas de retenção. A coexistência de múltiplos componentes \textit{open-source} com diferentes ciclos de maturidade, destacando-se a componente de \textit{logs}, que ainda depende de integrações externas como Loki ou ELK, implica um esforço contínuo de manutenção e atualização. Acresce que nem todos os serviços da plataforma R2UT se encontram ainda migrados para Kubernetes, limitando temporariamente a visibilidade transversal e a correlação total de telemetria, embora tal limitação esteja a ser mitigada pelo plano de migração progressiva em curso. Por fim, a eficácia da solução depende da literacia técnica da equipa e da adoção de processos operacionais maduros para análise e atuação sobre os indicadores de monitorização. Ainda assim, os benefícios práticos obtidos superam largamente estas limitações, demonstrando a robustez e a aplicabilidade desta abordagem em ambientes \textit{cloud-native}.

Em suma, neste trabalho conseguimos demonstrar que é possível construir um sistema de observabilidade robusto e eficiente em um ambiente \textit{cloud-native} utilizando ferramentas open-source. Essa abordagem não apenas melhora a confiabilidade e o desempenho da aplicação, mas também capacita as equipas a tomar decisões e a inovar com mais agilidade.



\section{Trabalho Futuro}

Embora a solução desenvolvida tenha demonstrado a sua eficácia na monitorização e correlação de métricas, \textit{logs} e \textit{traces} em ambiente \textit{cloud-native}, existem diversas oportunidades de evolução e consolidação que poderão potenciar significativamente o seu impacto operacional e científico.

Uma primeira linha de desenvolvimento consiste na \textbf{integração completa da telemetria em toda a plataforma R2UT}, acompanhando o plano de migração gradual para Kubernetes. A plena instrumentação de todos os serviços permitirá alcançar visibilidade ponta-a-ponta e uma correlação completa entre os fluxos de execução, tornando possível uma análise operacional verdadeiramente holística. Este esforço deverá incluir não apenas a instrumentação de novas APIs, mas também a definição de \textit{standards} internos formais para telemetria, garantindo consistência semântica e metodológica em toda a plataforma.

Outra vertente relevante consiste na \textbf{exploração da instrumentação automática (\textit{zero-code instrumentation})}, aproveitando os avanços contínuos na comunidade OpenTelemetry. Embora tenham sido identificadas limitações nesta implementação inicial, espera-se que a evolução do suporte nativo para .NET e a maturação dos agentes de instrumentação possam permitir a adoção parcial ou total desta abordagem no futuro, reduzindo o esforço de manutenção e eliminando a necessidade de inclusão de bibliotecas adicionais nos serviços.

A implementação de \textbf{dashboards avançados e automatização do \textit{alerting}} constitui também uma área de expansão crítica. A definição de \textit{Service Level Indicators} (SLI) e \textit{Service Level Objectives} (SLO) formais permitirá monitorizar níveis de serviço associados à disponibilidade, latência e fiabilidade, alinhando a observabilidade com a governança de serviço e as metas operacionais da plataforma. Complementarmente, a configuração de estratégias de alerta baseadas em anomalias e tendências (em vez de simples \textit{thresholds}) contribuirá para uma resposta mais proativa a falhas.

Outra frente de investigação com elevado potencial consiste na \textbf{integração de técnicas de Inteligência Artificial e Machine Learning} para deteção preditiva de anomalias e suporte à tomada de decisão. A análise automática de padrões de degradação e a implementação de mecanismos de correlação assistida poderão reduzir tempos de diagnóstico, potenciar respostas automáticas e aproximar a plataforma de um modelo de operação AIOps.

A adoção de mecanismos de \textbf{tracing unificado em ambientes híbridos e de múltiplos clusters} representa também um caminho promissor, possibilitando a observação integrada de componentes que residam em infraestruturas distintas (por exemplo, máquinas físicas, serviços externos e múltiplos clusters Kubernetes). Esta abordagem pode ser suportada por \textit{OpenTelemetry Gateways}, arquiteturas \textit{mesh} de observabilidade e mecanismos de encriptação e roteamento seguro de telemetria.

Por fim, uma linha de evolução natural reside na \textbf{automação do ciclo de vida da observabilidade com práticas GitOps}, garantindo que configurações do OpenTelemetry Collector, dashboards Grafana e regras de alerta são versionadas, auditáveis e sincronizadas com o estado desejado do cluster. Tal abordagem reforçará a consistência, segurança e fiabilidade do sistema ao longo do tempo.

Em suma, o trabalho futuro deverá focar-se na expansão da cobertura, no aumento do grau de automação, na integração de capacidades inteligentes e na consolidação de práticas de observabilidade como parte integrante do ciclo de vida de desenvolvimento e operação da plataforma. Ao perseguir estas linhas de evolução, será possível não só ampliar o valor prático desta solução no contexto do R2UT, como também contribuir para o avanço do estado da arte em observabilidade aplicada a sistemas distribuídos e ambientes industriais.
