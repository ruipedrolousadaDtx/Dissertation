\chapter*{Resumo}

As arquiteturas de microsserviços permitem construir sistemas flexíveis, escaláveis e modulares em ambientes distribuídos. Contudo, a sua natureza dinâmica aumenta a complexidade dos processos de monitorização contínua, deteção de falhas e resposta proativa a eventos críticos.
Neste trabalho de dissertação, foi implementada uma plataforma para a monitorização e gestão de alertas de infraestruturas baseadas em microsserviços, com aplicação prática na indústria da construção modular. A solução desenvolvida integra ferramentas \textit{open source} — \textit{Prometheus}, \textit{Grafana}, \textit{Loki} (como alternativa à \textit{ELK Stack}) e \textit{Jaeger} — suportadas pelo \textit{OpenTelemetry} para a recolha padronizada de métricas, \textit{logs} e rastreio (\textit{tracing}) distribuído.
Para além disso, foram definidos cenários de alerta e mecanismos de resposta automática, de forma a reforçar a resiliência e reduzir o tempo de indisponibilidade da infraestrutura em monitorização.
Os resultados obtidos demonstram ganhos significativos em visibilidade ponta-a-ponta e uma redução nos tempos médios de deteção e resolução de incidentes (\textit{MTTD} e \textit{MTTR}), comprovando a viabilidade de uma pilha de observabilidade aberta, escalável e alinhada com requisitos de produção.


\paragraph{Palavras-chave:} Arquitetura de Microsserviços, Contentorização, Kubernetes, Monitorização de Sistemas Distribuídos, Docker, Deteção e Resolução de Falhas.


\cleardoublepage

\chapter*{Abstract}

Microservices architectures enable the development of flexible, scalable, and modular systems in distributed environments. However, their dynamic nature increases the complexity of continuous monitoring, fault detection, and proactive response to critical events.
This dissertation presents the implementation of a monitoring and alert management platform for microservices-based infrastructures, with practical application in the modular construction industry. The proposed solution integrates \textit{open source} tools - \textit{Prometheus}, \textit{Grafana}, \textit{Loki} (as an alternative to the \textit{ELK Stack}), and \textit{Jaeger} - supported by \textit{OpenTelemetry} for standardized collection of metrics, \textit{logs}, and distributed \textit{tracing}.
Furthermore, alert scenarios and automated response mechanisms were defined to strengthen system resilience and reduce infrastructure downtime.
The results demonstrate significant gains in end-to-end visibility and a reduction in the mean time to detect and resolve incidents (\textit{MTTD} and \textit{MTTR}), validating the feasibility of an open, scalable, and production-ready observability stack.



\paragraph{Keywords} Microservices Architecture, Containerization, Kubernetes, Distributed Systems Monitoring, Docker, Fault Detection and Troubleshooting.

\cleardoublepage
